\begin{frame}[fragile]
    \autotitle
    \href
        {https://github.com/openshift/ci-tools/blob/master/cmd/ci-operator/main.go}
        {cmd/ci-operator/main.go}
    \footnotesize
    \begin{verbatim}
steps, postSteps, err := defaults.FromConfig(
    ctx, o.configSpec, &o.graphConfig, o.jobSpec,
    o.templates, o.writeParams, o.promote, o.clusterConfig,
    leaseClient, o.targets.values, o.cloneAuthConfig,
    o.pullSecret, o.pushSecret, o.censor, o.hiveKubeconfig,
    o.consoleHost, o.nodeName)
    \end{verbatim}
    \note{
        Moving on to what actually happens when \texttt{ci-operator} is
        executed, we start with this gigantic function call to turn all input
        values into a list of steps, still unordered.
    }
\end{frame}

\begin{frame}[fragile]
    \autotitle
    \href
        {https://github.com/openshift/ci-tools/blob/master/pkg/api/graph.go}
        {pkg/api/graph.go} (simplified)
    \begin{verbatim}
type InputDefinition []string

type Step interface {
    …
    Inputs() (InputDefinition, error)
    …
}
    \end{verbatim}
    \note{
        Recall that each step has an \texttt{Inputs} method, which returns a
        list of strings based on the values it depends on --- input image tags,
        source code revision, etc.
    }
\end{frame}

\begin{frame}[fragile]
    \autotitle
    \href
        {https://github.com/openshift/ci-tools/blob/master/cmd/ci-operator/main.go}
        {cmd/ci-operator/main.go} (simplified)
    \footnotesize
    \begin{verbatim}
func (o *options) resolveInputs(steps []api.Step) {
    var inputs api.InputDefinition
    for _, step := range steps {
        inputs = append(inputs, step.Inputs()...)
    }
    inputs = append(inputs, string(o.configSpec))
    inputs = append(inputs, o.extraInputHash.values...)
    stat := os.Stat(exec.LookPath(os.Args[0]))
    inputs = append(inputs, fmt.Sprintf(
        "%d-%d", stat.ModTime().UTC().Unix(), stat.Size()))
    sort.Strings(inputs)
    o.inputHash = inputHash(inputs)
    if len(o.namespace) == 0 {
        o.namespace = "ci-op-{id}"
    }
    o.namespace = strings.Replace(
        o.namespace, "{id}", o.inputHash, -1)
}
    \end{verbatim}
    \note{
        These values, along with those derived from the inputs such as the
        version and configuration file of \texttt{ci-operator}, are combined and
        hashed, generating a final string which is unique for a given set of
        inputs.

        This string is then used as a namespace, guaranteeing that:
        \begin{itemize}
            \item unrelated jobs are isolated from each other
            \item related jobs share resources as much as possible
        \end{itemize}
    }
\end{frame}

\begin{frame}[fragile]
    \autotitle
    \href
        {https://github.com/openshift/ci-tools/blob/master/cmd/ci-operator/main.go}
        {cmd/ci-operator/main.go} (simplified)
    \small
    \begin{verbatim}
var encoding = base32
    .NewEncoding("bcdfghijklmnpqrstvwxyz0123456789")
    .WithPadding(base32.NoPadding)

func inputHash(inputs api.InputDefinition) string {
    hash := sha256.New()
    for _, s := range inputs {
        hash.Write([]byte(s))
    }
    return encoding.EncodeToString(hash.Sum(nil)[:5])
}
    \end{verbatim}
    \note{Here is the (very simple) hash calculation.}
\end{frame}

\begin{frame}[fragile]
    \autotitle
    \href
        {https://github.com/openshift/ci-tools/blob/master/cmd/ci-operator/main.go}
        {cmd/ci-operator/main.go} (\emph{very} simplified)
    \small
    \begin{verbatim}
steps, postSteps := defaults.FromConfig(…)
o.resolveInputs(steps)
nodes := api.BuildPartialGraph(steps, o.targets.values)
stepList := nodes.TopologicalSort()
logrus.Infof(
    "Running %s",
    strings.Join(nodeNames(stepList), ", "))
o.initializeNamespace()
steps.Run(ctx, nodes)
for _, step := range postSteps {
    runStep(ctx, step)
}
    \end{verbatim}
    \note{
        Next, the graph is built by creating edges based on the dependency
        relations between steps (e.g. a test requires its container image to be
        imported or built).  Then, the test namespace is initialized and,
        finally, the steps are executed.

        So here is our final approximation of a full \texttt{ci-operator}
        execution.
    }
\end{frame}
